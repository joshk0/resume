\documentclass[overlapped,line,margin]{res}
\usepackage[colorlinks]{hyperref}
\usepackage{fontawesome5}
\begin{document}

\name{Joshua M. Kwan \tt{<joshk@triplehelix.org>}}
\address{
  \faIcon{globe} Queens, NY, USA
  \faIcon{phone} +1 (510) 646-0724
  \faIcon{github} \href{https://github.com/joshk0}{joshk0}
  \faIcon{linkedin} \href{https://linkedin.com/in/joshk0}{joshk0}
  \faIcon{calendar} last updated \href{https://circleci.com/gh/joshk0/resume}{\today}
}

\begin{resume}

\section{OBJECTIVE}
To contribute my 15 years of experience in the software industry to
build excellent products operating on top of robust infrastructure.
To maximize what I learn every day from my peers both within and outside
of the software realm.
To apprentice and mentor the next generation of software engineers with
empathy and understanding.

\section{EDUCATION} \textit{Bachelor of Science,} Electrical Engineering and Computer Science \\
  University of California, Berkeley, CA, December 2008

\section{SKILLS}
\begin{itemize}
\item \textit{Programming Languages:} 
  Go, Python, Makefile, Bash scripting, Ruby, C, C++, SQL, Java, JavaScript,
  Autoconf, C\#, HTML, CSS, \LaTeX, MIPS assembly, Lisp/Scheme, PHP
\item \textit{Operating systems:}
  Proficient at using, administrating, and developing on modern Unix (Linux,
  OS X, Solaris, FreeBSD) and Windows platforms
\item \textit{Frameworks, Libraries and Platforms:}
  Google Cloud Platform, Kubernetes, Docker, Amazon Web Services, Chef,
  GnuPG, OpenSSL, Android, POSIX, REST, Win32 API, cURL, Glib, GTK+, WiX/MSI
\item \textit{Source Control Systems:}
  Git, Perforce, Subversion, CVS, Arch/Bazaar, Visual SourceSafe
\item \textit{Spoken Languages:}
  Native speaker and writer of English and French,
  basic spoken/written knowledge of Japanese and Mandarin.
\item Strong system and network administration skills for Linux and a
  deep understanding of Linux distribution design from the ground up, from
  kernel to packaging system.
\item Ability to design and implement cloud infrastructure for large
  organizations with strong security and governance. Experienced in
  on-prem to cloud migrations.
\item \href{http://bcert.me/sfcqbkgtf}{Certified Scrum Product Owner}
\end{itemize}

\section{EXPERIENCE}
\textit{Senior Software Engineer, DevOps} \hfill November 2016 - \textit{present} \\
\textbf{LiveRamp Holdings, New York, NY (formerly Acxiom Corporation)}

  After Arbor was acquired by LiveRamp, led and managed 6+ engineers in
  the DevOps team. Successfully led this team over 18 months to fully
  migrate LiveRamp's environment from on-prem datacenter to Google Cloud
  Platform. The migration of 90PB of data and 90,000 cores of data
  processing workloads began in June of 2018 and concluded in November
  2019 with zero perceptible downtime by customers.

  Succeeded in improving much of LiveRamp's SDLC during the process by
  "lifting and improving" our existing Chef and VMware environment to
  Docker and Kubernetes. Worked cross-functionally to implement improved
  security, governance and cost control processes compared to the
  datacenter as well.  Mediated between security, compliance, and
  software development teams to the right balance of developer
  ownership, risk avoidance, and speed of product delivery.

\textit{Senior Software Engineer} \hfill May 2015 - November 2016 \\
\textbf{Arbor Technologies, New York, NY}

  First technical non-founder at small digital advertising startup.
  Provided leadership and ownership in the design of a data platform for
  receiving, storing, and distributing customer data. Webscaled, highly
  available Go codebase deployment on Google Cloud Platform receives
  over 10,000 hits per second.

\textit{Senior Infrastructure Engineer} \hfill January 2014 - May 2015 \\
\textbf{Schrodinger, Inc., New York, NY}

  Scaled cloud infrastructure (Amazon Web Services and VMware vSphere) to
  support LiveDesign, a unique collaborative drug design product. Helping with
  the logistical challenges of bringing a 10 year old Java codebase up to date
  with the realities of modern software while continuing to innovate on top of
  it. Automated deployments of LiveDesign using Chef on various flavors of
  Linux.

\textit{Member / Senior Member of Technical Staff} \hfill January 2009 - December 2013 \\
\textbf{VMware, Inc., Palo Alto, CA}

  \textbf{Horizon View HTML Access:} HTML Access provides browser-based
  connectivity to a Horizon View virtual desktop. Leveraged HTML5 to
  provide a smooth desktop experience even under network load and
  latency across most modern desktop and mobile browsers. Wrote RESTful
  APIs, dug into Windows' Terminal Services APIs, and maintained several
  installers surrounding the feature.

  \textbf{Horizon View Seamless Windows:} A port (80\% reuse) of the
  `Unity' feature from VMware's consumer products to Horizon View.
  Seamless Windows removes the background from a user's View virtual
  desktop so that the Windows appear seamlessly alongside the user's
  other applications. They can be resized and dragged natively. The
  port involved challenges such as cross-browser functionality and
  higher latencies over the network.

  \textbf{ThinApp Factory:} Developed a product to help automate
  creation of ThinApp packages (virtualized applications for Windows) at
  a large scale. Helped build a virtual appliance which used Debian,
  Java and the vSphere API to create and maintain ThinApps using a
  continuous integration process. The product has since been
  open-sourced and is available at
  \faIcon{github}
  \href{https://github.com/vmware/thinapp\_factory}{vmware/thinapp\_factory}.

  \textbf{Component Download Service (CDS):} Co-designed and implemented
  client and server APIs in C which allow delivery of software updates for
  all VMware products, built on top of cURL and OpenSSL. Added specific
  functionality for Windows, Linux and OS X. Products using CDS were
  first relesed in October 2009 and the framework is still used today
  (as of January 2020.)

\textit{Network Manager} \hfill August 2007 - January 2008 \\
\textbf{Berkeley Student Cooperative, Berkeley, CA}

  Managed internal network infrastructure, firewall, routing
  software, Internet connection, and file servers for a house of 120 residents.
  Responsibilities include tasks from low-level (reinstalling broken Ethernet
  ports) to high-level (managing network QoS and traffic-shaping in Linux.)

\textit{Intern} \hfill Summer 2007 \\
\textbf{VMware, Inc., Palo Alto, CA}

  \textbf{Linux Easy Install}: Improved feature for VMware consumer
  products which enables unattended installation of Linux distributions
  from ISOs and CD media. This was important for users with slower
  internet connections which were common at the time. Added support for
  Ubuntu and Red Hat-based distributions during a 2-month internship.
  Continued working on this after I became full-time and added support
  for SuSE Linux.

\textit{Volunteer Developer} \hfill 2003 - \textit{present} \\
\textbf{\href{https://www.debian.org}{The Debian Project}, worldwide}

  Assembled and maintained software packages for the Debian GNU/Linux
  distribution, involving knowledge of Make, shell, C, and Perl. Added
  the ability to use a wireless adapter while installing Debian (netcfg
  subsystem written in C.)

\end{resume}
\end{document}
