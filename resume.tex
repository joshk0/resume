% LaTeX resume using res.cls
\documentclass[overlapped,line,margin]{res}
%\usepackage{helvetica} % uses helvetica postscript font (download helvetica.sty)
%\usepackage{newcent}   % uses new century schoolbook postscript font

\begin{document}

\name{Joshua M. Kwan \tt{<joshua.m.kwan@gmail.com>}}
% \address used twice to have two lines of address
\address{(510) 646-0724}

\begin{resume}

\section{OBJECTIVE}
Seeking the opportunity to contribute my 15 years of experience in the open
source ecosystem to develop robust products using elegant combinations of
simple and non-overlapping components.

\section{EDUCATION} {\sl Bachelor of Science,} Electrical Engineering and Computer Science \\
  University of California, Berkeley, CA, December 2008

\section{SKILLS}
\begin{itemize}
\item {\sl Programming Languages:}
  C, C++, Go, Python, Makefile, Bash scripting, Ruby, SQL, Java, JavaScript,
  Autoconf, C\#, HTML, MIPS assembly, Lisp/Scheme, PHP
\item {\sl Operating systems:}
  Proficient at using, administrating, and developing on modern Unix (Linux,
  Solaris, FreeBSD, OS X) and Windows platforms
\item {\sl Frameworks, Libraries and Platforms:}
  Amazon Web Services, Google Compute Engine, Glib, GTK+, Win32 API, POSIX,
  cURL, REST, pthreads, WiX/MSI, Core Foundation, Qt, J2EE, Android, Chef,
  Django
\item {\sl Source Control Systems:}
  Git, Perforce, Subversion, CVS
\item {\sl Spoken Languages:}
  Native speaker and writer of English and French, basic spoken/written knowledge of
  Japanese.
\item Strong system and network administration skills for Linux and a
  deep understanding of Linux distribution design from the ground up, from
  kernel to packaging system.
\item Certified Scrum Product Owner (November 2017)
\end{itemize}

\section{EXPERIENCE}
{\sl Senior Engineering Manager, DevOps} \hfill November 2016 - Present \\
\textbf{LiveRamp, an Acxiom Company, New York, NY}

  After the acquisition of Arbor Technologies, the journey continues under
  the LiveRamp banner. Leading an 8-developer team spanning New York
  and San Francisco offices. The team's charter is to take LiveRamp's
  engineering org on a journey from a traditional VMware/CentOS
  on-premise environment into a AWS/Docker/Kubernetes deployment stack.

{\sl Senior Software Engineer} \hfill May 2015 - November 2016 \\
\textbf{Arbor Technologines, New York, NY}

  First technical non-founder at small digital advertising startup. Providing
  leadership and ownership in the design of a data platform for receiving,
  storing, and distributing customer data. Webscaled, fully redundant
  deployment receives over 10,000 hits per second and growing. Written using Go
  and deployed on Google Compute Engine.

{\sl Senior Infrastructure Engineer} \hfill January 2014 - May 2015 \\
\textbf{Schrodinger, Inc., New York, NY}

  Scaling cloud infrastructure (Amazon Web Services and VMware vSphere) to
  support LiveDesign, a unique collaborative drug design product. Helping with
  the logistical challenges of bringing a 10 year old Java codebase up to date
  with the realities of modern software while continuing to innovate on top of
  it. Automating deployments of LiveDesign using Chef on various flavors of
  Linux.

{\sl Senior Member of Technical Staff} \hfill May 2012 - December 2013 \\
\textbf{VMware, Inc., Palo Alto, CA}

  \textbf{Horizon View Seamless Windows:} Porting the 'Unity' feature
  from VMware Fusion and Workstation to work with VMware's Horizon View product.
  'Unity' lets a user view the windows inside a VM seamlessly next to the other
  windows on the VM host. Our port of the feature to remote desktops presents
  unique challenges such as network latency. We ported the existing feature to
  take advantage of code that has been tested in the field for over 5 years.
  Judicious refactoring and cooperation with the Unity team allowed us to reuse
  80\% of this code rather than needing to write our own solution from scratch.

  \textbf{Horizon View HTML Access:} HTML Access provides browser-based
  connectivity to a Horizon View virtual desktop. It leverages HTML5 to provide
  a smooth desktop experience even under network load and latency across most
  modern browsers, even those on mobile devices. I wrote RESTful APIs, dug into
  Windows' Terminal Services APIs, and maintained several installers
  surrounding the feature. The HTML Access feature has helped close many sales
  of Horizon View and in some cases our customers opt to use it over deploying
  the traditional View Client to their infrastructure.

{\sl Member of Technical Staff} \hfill January 2009 - April 2012 \\
\textbf{VMware, Inc., Palo Alto, CA}

  \textbf{ThinApp Factory:} Developed solutions to automate creation of
  ThinApp packages (virtualized applications for Windows) on a large scale
  through use of RSS feeds and a work pool hosted on virtual infrastructure.
  Helped build a turnkey appliance based on Debian that allows IT
  administrators to hit the ground running with our software. Using
  optimizations involving linked-clone virtual machines and snapshots, an admin
  can create Office 2010 as a ThinApp package within 90 minutes of deployment.

  \textbf{Component Download Service (CDS):} With my team, designed and wrote
  both client, server, and consumer API for a generic web updater framework for
  all VMware products, implemented in C using many open source libraries (cURL,
  c-ares, Glib, and OpenSSL.) The framework leverages platform-specific
  installer backends on Windows, Linux, and OS X, while providing a single
  unified API. This project made its debut in VMware Workstation 7.0 and VMware
  Fusion 3.0 and is still being used today.

{\sl Network Manager} \hfill August 2007 - January 2008 \\
\textbf{Berkeley Student Cooperative, Berkeley, CA}

  Managed internal network infrastructure, firewall, routing
  software, Internet connection, and file servers for a house of 120 residents.
  Responsibilities include tasks from low-level (reinstalling broken Ethernet
  ports) to high-level (managing network QoS and traffic-shaping in Linux.)

{\sl Intern} \hfill Summer 2007 \\
\textbf{VMware, Inc., Palo Alto, CA}

  \textbf{Linux Easy Install}: A feature for VMware products that enables people
  to simply insert an installation CD for a Linux distribution and instantly
  create a virtual machine containing that operating system without any hassle.
  Instead of forcing the user to download a hefty full virtual machine image,
  the distribution's unattended installation mechanisms are leveraged to allow
  quick deployment. Ubuntu and Red Hat-based distributions were the focus of
  this effort, and the system now supports SuSE Linux as well.

{\sl Volunteer Developer} \hfill 2003 - present \\
\textbf{The Debian Project (www.debian.org)}

  Putting together software packages for the Debian GNU/Linux distribution,
  involving knowledge of Make, shell, C, and Perl, and understanding OS
  infrastructure written in those languages. Also collaborated in writing the
  Debian Installer's network interface management code and introduced support
  for wireless configuration.

\end{resume}
\end{document}
